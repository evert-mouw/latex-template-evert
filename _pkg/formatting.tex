%% use geometry OR fullpage to decrease the page margins
%\usepackage[top=2cm,bottom=2cm,left=3cm,right=2cm]{geometry}
%% sets all 4 margins to be either 1 inch or 1.5 cm, and specifies the page style.
\usepackage{fullpage}

% Make a quote environment which is italicized.
% http://tex.stackexchange.com/questions/14683/defining-environments-based-on-other-ones-whats-the-right-way
\newenvironment{italicquote}{%
  \quote
  \itshape
}{%
  \endquote
}

%% **** MEMOIR ****
%% for best formatting / layout control, swith to memoir
%% in the index.tex, change the first line to the memoir class
%% disable the fullpage above and some of the things below
%% you can use a build-in memoir theme like pedersen:
%\chapterstyle{pedersen}

%% fancy headers
%% http://texblog.wordpress.com/2007/11/07/headerfooter-in-latex-with-fancyhdr/
%% Better choice would be to switch to memoir
%%  which has a much better pagestyle support than fancyhdr and book.
%% DON'T USE FANCYHDR WITH FULLPAGE, USE GEOMETRY PACKAGE INSTEAD
%% (which adds complexity)
%\usepackage{fancyhdr}
%\pagestyle{fancy}

%% line spacing if \linespread does not suffice
%% this adds three new environments: doublespace, onehalfspace, singlespace
\usepackage{setspace}

%% suppresses widows and orphans
\widowpenalty=10000 %% prevent single line of a paragraph (called "widow") remaining on the top of the succeeding page
\clubpenalty=10000 %% prevent single line of a paragraph remaining on the bottom of the preceding page

%% vertical alignment and paragraph spacing - use either raggedbottom or flushbottom
\raggedbottom %% FIXED vertical whitespace between paragraphs, but total heigt NOT the same on all pages
%\flushbottom %% VARIABLE vertical whitespace between paragraphs, but total heigt the same on all pages

%% \frenchspacing - no extra space after a period
%% Tells LATEX not to insert more space after a period than after ordinary character.
%% This is very common in non-English languages, except bibliographies.
%% If you use \frenchspacing, the command \@ is not necessary.
%\frenchspacing

%% endnote support, for example \endnote{text here}
%% where you want the endnotes to appear, type: \theendnotes
%% This will create .ent files, sometimes errors will happen (.ent file not found)
%\usepackage{endnotes}

%% Control line space and formatting of lists (enumerate, itemize, description)
%% labelindent=\parindent, leftmargin=*, itemsep=0pt, parsep=0pt
%% The "shortlabels" option emulates enumerate-like syntax, where A, a, I, i
%% and 1 stand for \Alph*, \alph*, \Roman*, \roman* and \arabic*. This is intended mainly as
%% a sort of compatibility mode with the enumerate package, and therefore the following special
%% rule applies: if the very frst option (at any level) is not recognized as a valid key, then it will
%% be considered a label with the enumerate-like syntax.
\usepackage[shortlabels]{enumitem}
\setlist{noitemsep}

%% Discourage hyphenation
%% http://dcwww.camd.dtu.dk/~schiotz/comp/LatexTips/LatexTips.html#nohyphen
%% penalties to things that don't look nice (such as words split over two lines)
%% The default penalty for splitting a word is rather low.
%% Increasing penalty will produce lines with a little more extra spacing between words,
%% and you should therefore increase TeX's tolerance for such lines.
%% adjusting \tolerance appears to be most promising
%% A \hyphenpenalty of 10000 (almost) prevents hyphenation, but produces overlong and/or ugly lines.
%\hyphenpenalty=5000
%\tolerance=1000

%% insert a pagebreak before a section
%% Gabriel Many, 2008-08
%% Evert Mouw, 2009-01
\makeatletter
\renewcommand{\section}{\@startsection
{section}
{1}
{0mm}
{-3.5ex \@plus -1ex \@minus -.2ex}
{2.3ex \@plus.2ex}
{\clearpage\phantomsection\normalfont\Large\bfseries}}
\makeatother{}

%% These two commands increase the space between two paragraphs
%% while setting the paragraph indent to zero.
%% but...
%% "parskip should not be used as it will also modify settings for 
%% list environments, table of contents, etc., and headings."
%% (An essential guide to LATEX2e usage - Obsolete commands and packages)
\setlength{\parindent}{0pt}
\setlength{\parskip}{1ex plus 0.5ex minus 0.2ex}

%% Elegant section numbering
%%
%% Pascal de Bruijn - The p-Code Machine - LaTeX tips & tricks
%% Source: http://blog.pcode.nl/2007/07/19/latex-tips-tricks/
%%
%% Modified by Evert Mouw to
%% - prefix the section symbol \S
%% - make it Gray; needs \usepackage[usenames,dvipsnames]{color}
%%
%% For my thesis I wanted the section numbering to be visible in the margins. This way the section titles
%% would be %% nicely lined up with my text, producing a very elegant look. Here is how you can do it:
\usepackage{sectsty}
\makeatletter\def\@seccntformat#1{\protect\makebox[0pt][r]{\textcolor{Gray}{\S \csname the#1\endcsname}\hspace{12pt}}}\makeatother
