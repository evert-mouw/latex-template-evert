%% sta toe om bijzondere letters zoals � direct te gebruiken,
%% en dat helpt ook meteen om een foutieve woordafbreking te voorkomen
%% [uitvoer] correcte uitvoer met goede woordafbreking rond speciale tekens, met eigen CM font:
\usepackage[T1]{fontenc}
%% [invoer] correcte invoer:
%\usepackage{ucs} % unicode
\usepackage[utf8x]{inputenc} % nix=latin1, dos=cp850, win=ansinew, mac=applemac, utf8 = unicode, utf8x = unicode extended (unofficial)

%% more unicode support
\usepackage{eurosym}
\usepackage{textcomp}

%% Make litagures searchable and copyable in PDF readers
\usepackage{cmap}
%%
%% For the Linux Libertine font this is not supported, you have to use:
%\input{glyphtounicode}
%\pdfglyphtounicode{f_f}{FB00}
%\pdfglyphtounicode{f_f_i}{FB03}
%\pdfglyphtounicode{f_f_l}{FB04}
%\pdfglyphtounicode{f_i}{FB01}

%% mooier font bij pdf export: CM en EC fonts
%% !! als het goed is, wordt dit al (deels) gedaan door de \usepackage[T1]{fontenc}
%\usepackage{ae,aecompl}
%\usepackage{aeguill}
%% of gebruik PostScript fonts (niet de eerste voorkeur)
%\usepackage{pslatex}
%% or use times, palatino, newcent, bookman, libertine;
%% also, have a look at http://www.tug.dk/FontCatalogue/
%\usepackage{times}
\usepackage{tgtermes} %% Extended times: works for MikTex, and needs "apt-get install tex-gyre" for Ubuntu with Texlive
%\usepackage{libertine} %% Libertine geeft problemen met \emph of \em, zoals nesting.
%\renewcommand*\oldstylenums[1]{{\fontfamily{fxlj}\selectfont #1}} %% Alleen voor Libertine, en niet zeker of dit nodig is!
